\noindent香港中文大學\\
計算機科學及工程學系哲學碩士\\
陳雋永

\vspace{1cm}

{
\setlength{\parindent}{2em}
很多現代存儲系統利用糾刪碼儲存少量冗餘數據,來提高系統的可用性。
日誌結構檔案系統把數據更新順序地寫入硬盤,來提高更新效率。可是,當
系統崩潰發生時,這種設計必須讀取分佈在不同地方的更新以重建數據塊和校驗塊,造成額外的讀寫成本。
我們提出的parity logging with reserved space有兩項主要設計特色︰(1)
結合in-place數據更新和log-based校驗更新,平衡更新和回復的效能。(2)把校驗更新放在
校驗塊旁的一個備用空間中,減少硬盤尋道的頻率。再者,我們提出一個按照系統負載
預測和調整備用空間的大小的方案,並應用在我們開發的一個糾刪碼分佈式存儲系統CodFS上。
利用CodFS,我們測試了不同更新方案在各種真實負載下的表現,發現普通的in-place
和log-based更新方案並不能像我們的方案般同時在更新和回復方面達到良好的性能。

除了分佈式存儲系統,隨機寫入也會對SSD RAID構成問題。在本論文的下半部分,我們
提出了一個以提高寫入性能和SSD壽命為目標的中間件TWEEN。TWEEN有兩項主要的設計特色︰(1)結合
log-structured文件系統和非揮發性記憶體(NVRAM),避免在SSD層面和RAID層面進行局部寫入。(2)組合
更新頻率類近的寫入和利用NVRAM吸收垃圾回收引致的寫入,減少LFS層面中的垃圾回收開銷。
}
