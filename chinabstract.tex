\noindent香港中文大學\\
計算機科學及工程學系哲學碩士\\
陳雋永

\noindent 基於負載感知設計的高性能糾刪碼存儲系統

\vspace{1cm}

很多現代存儲系統利用糾刪碼來提高系統的可用性,同時達到低冗餘數據的效果。
Log-structured文件系統把數據更新順序地寫入硬盤,來提高更新效率。可是,這個
設計當發生系統崩潰而需要回復時,會利用數據更新重建數據塊和校驗塊,造成顯著的讀寫開銷。
我們提出的parity logging with reserved space有兩項主要的設計特色︰(1)
結合in-place數據更新和log-based校驗更新,平衡更新和回復的開銷。(2)把校驗更新放在
校驗塊旁的一個備用空間中,減少硬盤尋道的開銷。我們進一步提出一個支持負載感知的方案,來動態
地預測和調整備用空間的大小。我們實現了一個糾刪碼分佈式存儲系統的原型,並稱之為CodFS。
在CodFS上,我們測試不同更新方案在各種真實負載下的表現,並發現普通的in-place
和log-based更新方案並不能同時在更新和回復方面達到良好的性能,只有我們的方案能達到這一個效果。

除了分佈式存儲系統,隨機寫入也會對SSD RAID構成問題。在本論文的下半部分,我們
提出了一個以提高寫入性能和SSD壽命為目的的中間件,並稱之為TWEEN。TWEEN有兩項主要的設計特色︰(1)結合
log-structured文件系統和非揮發性記憶體(NVRAM),避免在SSD層面和RAID層面的局部寫入。(2)透過把
更新頻率類近的寫入組合和利用NVRAM吸收垃圾回收寫入兩項技術,達到減少LFS垃圾回收開銷的效果。
