Many modern storage systems adopt erasure coding to provide data availability
guarantees with low redundancy.  Log-based storage is often used to
append new data rather than overwrite existing data so as to achieve high
update efficiency, but introduces significant \mbox{I/O} overhead during recovery due
to reassembling updates from data and parity chunks.  We propose parity
logging with reserved space, which comprises two key design features: (1) it
takes a hybrid of in-place data updates and log-based parity updates to
balance the costs of updates and recovery, and (2) it keeps parity updates in
a reserved space next to the parity chunk to mitigate disk seeks.  We further
propose a workload-aware scheme to dynamically predict and adjust the reserved
space size.  We prototype an erasure-coded clustered storage system called
CodFS, and conduct testbed experiments on different
update schemes under synthetic and real-world workloads.  We show that
our proposed update scheme achieves high update and recovery
performance, which cannot be simultaneously achieved by pure in-place or
log-based update schemes.

In addition to clustered storage, we also study the issues introduced by small random 
writes to SSD RAID. In the second part of this thesis, 
we propose TWEEN, a middleware application that aims toward write efficiency 
and endurance of SSD RAID.  It comprises two key design features: 
(1) it combines the log-structured file system (LFS) and byte-addressable 
non-volatile RAM (NVRAM) to eliminate partial writes at both SSD and RAID 
levels, and (2) it mitigates the garbage collection overhead in LFS by 
grouping writes with similar update frequencies and absorbing LFS 
garbage collection writes in NVRAM. 
