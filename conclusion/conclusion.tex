\chapter{Conclusions}
\label{chap:conclusions}

\section{Summary}

In CodFS, our key contribution is the parity logging with reserved space (\PLR) scheme,
which keeps parity updates next to the parity chunk to mitigate disk seeks. We
also propose a workload-aware scheme to predict and adjust the reserved space
size.  We build the CodFS prototype, which is an erasure-coded clustered storage system
that achieves efficient updates and recovery.  We evaluate CodFS 
using both synthetic and real-world traces and show that \PLR improves update
and recovery performance over pure in-place and log-based updates.  

We also propose TWEEN, a user-space middleware application designed to enhance
the write efficiency and endurance of SSD RAID. TWEEN utilizes the emerging
NVRAM technology as non-volatile cache and works seamlessly with off-the-shelf
SSD RAID solutions. It combines the log-structured file system (LFS) and NVRAM
to batch incoming partial writes and flush them to SSD RAID as full-stripe
writes. It also exploits workload characteristics to reduce LFS garbage
collection overhead. 

\section{Future Work}

\subsection{Applying CodFS on Other Storage Media}

Currently, CodFS optimizes update performance on disk-based clustered storage
with commodity hardware configurations. We may extend CodFS to other storage
media, such as those utilizing DRAM for storage \cite{ongaro11}, or those using
a combination of NVRAM and SSDs \cite{qiu13}. Although the seek time may not be
as significant in these types of storage media, we believe CodFS can still
benefit performance by reducing fragmentation in the system. However, an
important limitation for these types of storage media is the high storage cost.
We pose the issue of further reducing the reserved space overhead as future
work.

\subsection{Exploring New Reserved Space Management Designs}

In our work, CodFS only uses a simple workload-aware design for reserved space
management. Although we show that this heuristic works reasonably well in
evaluations, we plan to explore more advanced designs. For example, instead of
shrinking the reserved space of all segments daily, it is beneficial to
periodically shrink some of the popular segments during idle time. To choose the
right segments to shrink, we may
utilize existing victim selection policies in LFS, such as cost-benefit
\cite{rosenblum92} and cost-hotness \cite{min12}.

\subsection{Deploying TWEEN on Single-SSD Systems}

Currently, TWEEN focuses on SSD RAID deployments, which may limit its
applications to enterprise environments.  However, since our system decouples
the RAID layer from the file system layer, it is also possible to adopt TWEEN on
a single-SSD setup. We believe that by batching small random writes in a
non-volatile buffer and aligning the granularity of the SSD garbage collection,
our design can also benefit the write performance and endurance in a desktop
environment using a single SSD.  We pose the evaluation of TWEEN on single-SSD
deployments under desktop workloads as future work.
