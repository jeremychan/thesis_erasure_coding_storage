\chapter{Conclusions}
\label{chap:conclusions}

In this chapter, we summarize our work and address future improvements for both
CodFS and TWEEN.

\section{Summary}

In CodFS, our key contribution is the parity logging with reserved space (\PLR) scheme,
which keeps parity updates next to the parity chunk to mitigate disk seeks. We
also propose a workload-aware scheme to predict and adjust the reserved space
size.  We build the CodFS prototype, which is an erasure-coded clustered storage system
that achieves efficient updates and recovery.  We evaluate CodFS 
using both synthetic and real-world traces and show that \PLR improves update
and recovery performance over pure in-place and log-based updates.  

We also propose TWEEN, a user-space middleware application designed to enhance
the write efficiency and endurance of SSD RAID. TWEEN utilizes the emerging
NVRAM technology as non-volatile cache and works seamlessly with off-the-shelf
SSD RAID solutions. It combines the log-structured file system (LFS) and NVRAM
to batch incoming partial writes and flush them to SSD RAID as full-stripe
writes. It also exploits workload characteristics to reduce LFS garbage
collection overhead. 

\section{Future Work}

\subsection{Latency Evaluation}

We plan to evaluate other metrics (e.g., latency) of different parity
update schemes. 

\subsection{Overhead of Shrinking and Merging}

We evaluate the impact of the shrinking and merging
operations on throughput and latency. 

\subsection{Reserved Space Management Design}

We explore a more robust design of
reserved space management.
