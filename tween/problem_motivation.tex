\section{Partial Writes in SSD RAID}
\label{sec:partial}

Small random writes are detrimental to both SSD performance
\cite{kim08,chen09,min12} and RAID performance \cite{stodolsky93}.  They exhibit
in many real-world workloads, such as those in online transaction-processing
(OLTP) systems \cite{wong02}, desktops \cite{harter11} and enterprise servers
\cite{kavalanekar08}. Also, modern databases and critical applications tend to
call \texttt{fsync/sync} frequently to force {\em synchronous writes}, which
cause writes to be relatively small and random \cite{harter11}.
	   
%partial-stripe writes frequently.  Moreover, recent studies show that random
%writes commonly occur in both enterprise \cite{chan14} and desktop
%\cite{harter11} workloads.  Also, 
	
This paper considers two specific types of small random writes,
namely {\em partial-block writes} and {\em partial-stripe writes}, whose
request sizes are too small and do not align with the granularities of the SSD
and RAID operations, respectively.  We collectively call them 
{\em partial writes}, which incur performance overhead to SSD RAID as
described below. 

%\subsection{Partial-Block Writes} 

%Partial writes introduce performance overhead to an SSD RAID.  We classify
%partial writes into \textit{partial-stripe writes} in RAID and
%\textit{partial-block updates} in SSDs. In RAID, partial-stripe writes refer to
%writes that cover only part of a stripe. While these writes are handled using
%either read-modify write or reconstruct write in a parity-based RAID, both
%require extra read operations for maintaining parity consistency.  In SSDs,
%\textit{partial-block updates} refer to writes that change a portion of pages
%in a block. Partial-block updates hinder garbage collection performance and
%shorten the lifetime of SSDs as they tend to fragment a block with live and
%stale pages.

%Figure~\ref{fig:typicalraid} shows a typical SSD RAID setup.  Partial writes
%can occur in either (or both) the (1) SSD layer and the (2) RAID layer.  Our
%goal is to eliminate both partial-stripe writes and partial-block updates
%\textit{on the write path} to achieve optimal update performance and enhance
%SSD lifetime.  
%In this section, we separately study the causes and impacts of the two kinds
%of partial writes and propose a novel file system layer that couples NVRAM
%with a log-structured design as solution.

%\paragraph{When do they happen} 

\subsection{Partial-Block Writes}

In the SSD layer, partial-block writes are the writes that update only a
portion of pages in an SSD block.  Because of out-of-place writes in SSDs, 
partial-block writes eventually cause SSD blocks to
contain a mix of live and stale pages, leading to internal fragmentation
\cite{chen09}.  Such internal fragmentation increases the write amplification
overhead of garbage collection, because each time garbage collection needs to
copy many live pages from the chosen block to a free block. If the number of
stale pages in a block is small, then live pages of multiple blocks need to be
relocated so as to reclaim a block that is full of clean pages. Note that the
large write amplification overhead is seen regardless of the underlying
address mapping schemes in FTL \cite{min12}. 

%In an SSD RAID, we see that partial-block updates occur if the RAID layer
%issues write requests that are either (1) unaligned to block boundaries or
%(2) having write size smaller than a block. Typically, the SSD controller
%handles partial-block updates by writing the new data to a clean block and
%marking the original pages as \textit{stale}.

%\paragraph{Why are they harmful} 

%The impacts of partial-block updates are two-fold. First, modern SSDs adopt
%interleaving techniques to boost I/O performance. Large writes spanning across
%multiple pages are inherently faster than small writes. Writes in
%multiples of the \textit{clustered page} \cite{kim12toc} size achieve better 
%I/O performance as they can write to multiple NAND flash chips in parallel.

%\red{[write size vs throughput experiment]}

%However, the fragmentation of live and stale pages in blocks eventually leads
%to write amplification. For page-level mapping FTLs, garbage collection
%suffers from large copying overhead as the controller needs to migrate live
%pages from the victim block to another clean block. For hybrid mapping FTLs,
%garbage collection is likely to trigger the expensive \textit{full merge}
%\cite{gupta09} which copies live pages from both the log blocks and data
%blocks. In either case, write amplification occurs as a result of the
%NAND flash chips receiving extra page writes due to partial-block
%updates.

%\red{[write amplification experiment]} \\

%\paragraph{Solution} 

%One intuitive approach to eliminate partial-block updates is to let the RAID
%layer align writes to the SSD block boundary before issuing them.  For
%page-level mapping FTLs, this eliminates copying overhead in garbage
%collection as all pages in a SSD block are either stale or live.  For hybrid
%mapping FTLs, garbage collection becomes a series of efficient \textit{switch
%merge} \cite{gupta09} operations since one can just replace a data block
%of stale pages with its corresponding log block.  However, the alignment
%operation is costly because it often enlarges the write size by first
%reading data to pack the unaligned writes.

\subsection{Partial-Stripe Writes}

%\paragraph{When do they happen} 

Partial-stripe writes are the writes whose request sizes are smaller than the
size of a stripe in RAID.  Depending on the write size, RAID handles a 
partial-stripe write based on one of the two approaches \cite{chen95}: (1)
\textit{read-modify write}, which reads the old data chunks to be updated and
all parity chunks of the same stripe, applies the difference between the old
and new data chunks to the parity chunks, and finally writes the new data and
parity chunks to devices, and (2) \textit{reconstruct write}, which reads all
non-updated data chunks in the stripe,  calculates the new parity chunks based
on the data chunks that are just read and the new data chunks to be written,
and finally writes all data and parity chunks within the stripe.  Both
reconstruct writes and read-modify writes are inferior in performance as they
incur extra reads before writes.  Instead, it is more desirable to issue 
{\em full-stripe writes}, which incur no read by updating all chunks in a stripe 
and are the most efficient type of writes in RAID \cite{chen95}. 

%Typically, workloads dominated by small and random writes such as those from
%online transaction-processing (OLTP) systems \cite{wong02} tend to trigger
%partial-stripe writes frequently.  Moreover, recent studies show that random
%writes commonly occur in both enterprise \cite{chan14} and desktop
%\cite{harter11} workloads.  Also, modern databases and critical applications
%tend to call \texttt{fsync/sync} frequently to force writes to disk, which
%cause writes to be relatively small and random \cite{harter11}.

%\paragraph{Why are they harmful} 

%\paragraph{Solution} 

%However, while this improves user experience, it only delays does not work well
%under intensive workload once the NVRAM is exhausted \red{[CITE]}.

%Intuitively, a log-based design converts small random writes to large
%sequential writes through batching. Updates are packed in a log until there
%are sufficient data to issue a full-stripe write~\cite{menon95}.  Although a
%log-based design eliminates read operations on the write path, the garbage
%collection mechanism requires careful consideration as it is known to hinder
%system performance~\cite{seltzer95}.

